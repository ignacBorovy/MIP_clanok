% Metódy inžinierskej práce

\documentclass[10pt,twoside,slovak,a4paper]{article}

\usepackage[slovak]{babel}
%\usepackage[T1]{fontenc}
\usepackage[IL2]{fontenc} % lepšia sadzba písmena Ľ než v T1
\usepackage[utf8]{inputenc}
\usepackage{graphicx}
\usepackage{url} % príkaz \url na formátovanie URL
\usepackage{hyperref} % odkazy v texte budú aktívne (pri niektorých triedach dokumentov spôsobuje posun textu)

\usepackage{cite}
%\usepackage{times}

\pagestyle{headings}

\title{Vývoj crossplatformových (OS) aplikácii\thanks{Semestrálny projekt v predmete Metódy inžinierskej práce, ak. rok 2015/16, vedenie: Vladimír Mlynárovič}}

\author{Ignác Borový\\[2pt]
	{\small Slovenská technická univerzita v Bratislave}\\
	{\small Fakulta informatiky a informačných technológií}\\
	{\small \texttt{...@stuba.sk}}
	}

\date{\small 19. october 2021}



\begin{document}

\maketitle


\section{Úvod}
\quad
Keďže tu máme rôzne typy zariadený mobily, tablety, počítače a atd.. Tak na tieto zariadenia existuje veľa rôznych operačných systémov ako je Windows, Linux, Android a ďalšie. Tu nám vzniká problém ako vyvinúť software tak aby fungoval na všetky tieto operačné systémy, keďže každý systém na inakšiu architektúru a nástroje. 

Riešením je crossplatformový vývoj. Zamerám sa hlavne na to ako sa technologie vysporiadavajú s týmto problémov prenositeľnosti. Čiže sa budem zaoberať ako aplikácia dokáže nezávisle fungovať od platformy, akú architektúru takéto aplikácie používajú a ako taký kód vo finále vyzerá.

\section{Prenositeľnosť}
\subsection{Výhody a nevýhody}

\quad
Softwerova prenositeľnosť ma rôzný význam pre rôzných ľudí. V našom prípade sa jedná o prenositeľnosť softwaru medzi rôznymi platformami v našom prípade OS. Najväčšia výhoda platformovej prenositeľnosti je že dokážeme väčšiu skupinu uživateľov ktorý používaju rôzne operačné sýstemy. 
Ďalšími výhodami sú:
\begin{itemize}
    \item menšia cena za vývoj aplikácie
    \item ľahší vývoj a ľahšie údržba aplikácie.
\end{itemize}
Nevýhody:
\begin{itemize}
    \item pomalšia rýchlosť aplikácie
    \item väčšia veľkosť aplikácie
\end{itemize}

\subsection{Typy riešenia prenositeľnosti}




\section{}

Niekedy treba uviesť zoznam:

\begin{itemize}
\item jedna vec
\item druhá vec
	\begin{itemize}
	\item x
	\item y
	\end{itemize}
\end{itemize}

Ten istý zoznam, len číslovaný:

\begin{enumerate}
\item jedna vec
\item druhá vec
	\begin{enumerate}
	\item x
	\item y
	\end{enumerate}
\end{enumerate}


\subsection{Ešte nejaké vysvetlenie} \label{ina:este}

\paragraph{Veľmi dôležitá poznámka.}
Niekedy je potrebné nadpisom označiť odsek. Text pokračuje hneď za nadpisom.



\section{Dôležitá časť} \label{dolezita}




\section{Ešte dôležitejšia časť} \label{dolezitejsia}



\section{Ešte dôležitejšia časť}



\section{Záver} \label{zaver} % prípadne iný variant názvu



%\acknowledgement{Ak niekomu chcete poďakovať\ldots}


% týmto sa generuje zoznam literatúry z obsahu súboru literatura.bib podľa toho, na čo sa v článku odkazujete
\bibliography{literatura}
\bibliographystyle{plain} % prípadne alpha, abbrv alebo hociktorý iný
\end{document}
